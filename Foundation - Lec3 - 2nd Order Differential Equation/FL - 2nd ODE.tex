\documentclass[12pt,a4paper]{article}
\usepackage[utf8]{inputenc}
\usepackage[T1]{fontenc}
\usepackage{amsmath}
\usepackage{amsfonts}
\usepackage{amssymb}
\usepackage{graphicx}
\usepackage[left=2.54cm, right=2.54cm, top=2.54cm, bottom=2.54cm]{geometry}
\usepackage[hidelinks]{hyperref} %for reference automatically
\usepackage{fancyhdr}

\usepackage{tikz}
\usetikzlibrary{positioning}

\pagestyle{fancy} % for header and footer
\fancyhf{}
\fancyhead[LE,RO]{Prepared by: Phayuth}
\fancyhead[RE,LO]{Supervisor : Dr.Sarot Srang}
\fancyfoot[LE,RO]{Page \thepage}
\renewcommand{\headrulewidth}{2pt}
\renewcommand{\footrulewidth}{1pt} % for header and footer

% THIS IS THE XML CODE INCLUDE==============================================================
\usepackage{listings}
\usepackage{xcolor}
\definecolor{codegreen}{rgb}{0,0.6,0}
\definecolor{codegray}{rgb}{0.5,0.5,0.5}
\definecolor{codepurple}{rgb}{0.58,0,0.82}
\definecolor{backcolour}{rgb}{0.95,0.95,0.92}
\lstset{
	backgroundcolor=\color{backcolour},   
	commentstyle=\color{codegreen},
	keywordstyle=\color{magenta},
	numberstyle=\tiny\color{codegray},
	stringstyle=\color{codepurple},
	numbers=left,
	breaklines=true,
	tabsize=2,
	basicstyle=\ttfamily\footnotesize,
	literate={\ \ }{{\ }}1
}


\begin{document}
	\section*{\centering Foundation - Lesson 3 : 2nd Order Differential Equation}
	%	\begin{itemize}
		%		\item {\makebox[1cm]{\(x(t)\)\hfill} is Input of the system}
		%		\item {\makebox[1cm]{\(y(t)\)\hfill} is Output of the system}
		%		\item {\makebox[1cm]{\(h(t)\)\hfill} is the system}
		%	\end{itemize}
	
	%	\begin{equation}
		%		\boxed{
			%			H(s) = \frac{Y(s)}{X(s)}
			%		}
		%		\label{eq3}
		%	\end{equation}
	\section{Background}
	Differential equation is set an equation that its solution is a function and involve of its derivative. In engineering, these equations is usually used to govern a dynamics system model and the rate of change of state. In 2nd Order Differential Equation is equation consist of second and first derivative of function in form:
	\[
	\frac{d^2y}{dx^2} + P(x)\frac{dy}{dx} + Q(x) y = f(x)
	\]
	\section{Method of Solving 2nd Order Homogeneous Differential Equation}
	The 2nd Order Homogeneous Differential Equation has the form of:
	\[
	\frac{d^2y}{dx^2} + P(x)\frac{dy}{dx} + Q(x) y = 0
	\]
	\subsection{Method of Using Characteristic Equation}
	\subsubsection{Where Does it come from ?}
	We propose a solution the 2nd Order Homogeneous Differential Equation above where:
	\[
	y = e^{rx}
	\]
	Thus
	\[
	\frac{d^2y}{dx^2}= r^2e^{rx}
	\]
	\[
	\frac{dy}{dx} = re^{rx}
	\]
	Substitute to the equation:
	\[
	\begin{split}
		r^2e^{rx} + P(x)re^{rx} + Q(x)e^{rx} &= 0 \\
		e^{rx} (r^2 + P(x)r + Q(x)) &= 0 \\
	\end{split}
	\]
	The term $e^{rx}$ can not go to zero, thus the term $r^2 + P(x)r + Q(x)$ will go to zero. The term $r^2 + P(x)r + Q(x)$ is the second order polynomial equation where we have three different form of solution.
	\begin{itemize}
		\item {\makebox[1cm]{\(\Delta > 0\)\hfill} the equation has 2 distinct real roots \(r_1\) and \(r_2\)}
		\item {\makebox[1cm]{\(\Delta = 0\)\hfill} the equation has repeated real root \(r\)}
		\item {\makebox[1cm]{\(\Delta < 0\)\hfill} the equation has 2 complex conjugated roots \(r_1 = \alpha + \beta i\) and \(r_2 = \alpha - \beta i\)}
	\end{itemize}
	If:
	\begin{itemize}
		\item {\makebox[1cm]{\(\Delta > 0\)\hfill} the solution has a form of \(y = Ae^{r_1x} + Be^{r_2x}\)}
		\item {\makebox[1cm]{\(\Delta = 0\)\hfill} the solution has a form of \(y = Ae^{rx} + Bxe^{rx}\)}
		\item {\makebox[1cm]{\(\Delta < 0\)\hfill} the solution has a form of}
			\begin{itemize}
				\item \(y = Ae^{(\alpha + \beta i)x} + Be^{(\alpha - \beta i)x}\)
				\item \(y = e^{\alpha x}(Ae^{\beta ix} + Be^{-\beta ix})\)
				\item \(y = e^{\alpha x}(A cos (\beta x) + iB sin (\beta x))\) (From euler's formula \(e^{ix} = cos(x)+isin(x)\))
			\end{itemize}
	\end{itemize}
	
	\subsubsection{How to use it}
	\begin{itemize}
		\item Step 1 : Form a Characteristic Equation from the equation \(ar^2 + br + c =0\).
		\item Step 2 : Find \(\Delta\) and roots of the equation.
		\item Step 3 : Plug the result into three of the solution form and determine the constants using initial condition.
	\end{itemize}
	
	\subsubsection{Example}
	1. Solve:
	\[
	\frac{d^2y}{dx^2} -9\frac{dy}{dx} + 20 y = 0
	\]
	\begin{itemize}
		\item Step 1
	\end{itemize}
	\[
	r^2 -9r + 20 =0
	\]
	\begin{itemize}
		\item Step 2
	\end{itemize}
	\[
	\Delta > 0, r_1 = 4, r_2 = 5
	\]
	\begin{itemize}
		\item Step 3
	\end{itemize}
	\[
	y = Ae^{4x} + Be^{5x}
	\]
	2. Solve:
	\[
	\frac{d^2y}{dx^2} -10\frac{dy}{dx} + 25 y = 0
	\]
	\begin{itemize}
		\item Step 1
	\end{itemize}
	\[
	r^2 -10r + 25 =0
	\]
	\begin{itemize}
		\item Step 2
	\end{itemize}
	\[
	\Delta = 0, r = 5
	\]
	\begin{itemize}
		\item Step 3
	\end{itemize}
	\[
	y = Ae^{5x} + Bxe^{5x}
	\]
	3. Solve:
	\[
	\frac{d^2y}{dx^2} -4\frac{dy}{dx} + 13 y = 0
	\]
	\begin{itemize}
		\item Step 1
	\end{itemize}
	\[
	r^2 -4r + 13 =0
	\]
	\begin{itemize}
		\item Step 2
	\end{itemize}
	\[
	\Delta < 0, r_1 = 2+3i, r_2 = 2-3i
	\]
	\begin{itemize}
		\item Step 3
	\end{itemize}
	\[
	y = e^{2 x}(A cos (3 x) + iB sin (3 x))
	\]
	
	
	
	
	\section{Method of Solving 2nd Order Non-Homogeneous Differential Equation}
	The 2nd Order Non-Homogeneous Differential Equation has the form of:
	\[
	\frac{d^2y}{dx^2} + P(x)\frac{dy}{dx} + Q(x) y = f(x)
	\]
	The solution of the 2nd Order Non-Homogeneous Differential Equation has a combination of General solution and Particular solution \(y = y_h + y_p\). The General solution is found by finding the solution of the equation in its homogeneous form while the Particular solution is found by finding the solution of the equation in its non-homogeneous.
	
	\subsection{Method of Undetermined Coefficients}
	\subsubsection{How to use it}
	\begin{itemize}
		\item Step 1 : Find \(y_h\) from the equation in homogeneous form.
		\item Step 2 : Propose \(y_p\) by guessing the form of solution from the non-homogeneous term. Use the table for help. And determine the constants of the \(y_p\).
		\item Step 3 : Find \(y = y_h + y_p\) and Find remaining constants from initial equation.
	\end{itemize}
	\subsubsection{Table of solution form}
	\begin{tabular}{|c|c|}
		\hline
		\(f(x)\)         & \(y_p\)                                 \\
		\hline
		1                & \(a\)                                   \\
		\hline
		\(5x+7\)         & \(ax+b\)                                \\
		\hline
		\(3x^2-2\)       & \(ax^2+b+c\)                            \\
		\hline
		\(x^3-x+1\)      & \(ax^3+bx^2+cx+d\)                      \\
		\hline
		\(sin4x\)        & \(acos4x+bsin4x\)                       \\
		\hline
		\(cos4x\)        & \(acos4x+bsin4x\)                       \\
		\hline
		\(e^{5x}\)       & \(ae^{5x}\)                             \\
		\hline
		\((9x-2)e^{5x}\) & \((ax+b)e^{5x}\)                        \\
		\hline
		\(x^2e^{5x}\)    & \((ax^2+b+c)e^{5x}\)                    \\
		\hline
		\(e^{3x}sin4x\)  & \(ae^{3x}cos4x+be^{3x}sin4x\)           \\
		\hline
		\(5x^2sin4x\)    & \((ax^2+b+c)cos4x+(dx^2+e+f)sin4x\)     \\
		\hline
		\(xe^{3x}cos4x\) & \((ax+b)e^{3x}cos4x+(cx+d)e^{3x}sin4x\) \\
		\hline
	\end{tabular}
	
	
	\subsubsection{Example}
	Solve:
	\[
	\frac{d^2y}{dx^2} - y= 2x^2 - x - 3
	\]
	\begin{itemize}
		\item Step 1
	\end{itemize}
	\[
	\begin{split}
		r^2 - 1 &= 0 \\
		\Delta > 0 &, r_1=1,r_2=-1 \\
		y_h &= Ae^{1x} + Be^{-1x}
	\end{split}
	\]
	\begin{itemize}
		\item Step 2
	\end{itemize}
	Let guess the form based on the \(2x^2 - x - 3\), because of it is a polynomial let guess \(y_p = ax^2 + bx + c\)
	\[
	\begin{split}
		\frac{dy_p}{dx} &= 2ax+b \\
		\frac{d^2y_p}{dx^2} &= 2a \\
	\end{split}
	\]
	Substitute back to equation
	\[
	\begin{split}
		2a - (ax^2 + bx + c) &= 2x^2 - x - 3 \\
		2a - ax^2 - bx - c &= 2x^2 - x - 3 \\
		-ax^2-bx+2a-c &=2x^2 - x - 3 \\
		-a &= 2 \\
		-b &= -1 \\
		2a-c &= -3 \\
		\rightarrow a &= -2 \\
		\rightarrow b &= 1 \\
		\rightarrow c &= -1 \\
	\end{split}
	\]
	Thus
	\[
	y_p = -2x^2 + x - 1
	\]
	\begin{itemize}
		\item Step 3
	\end{itemize}
	\[
	y =  Ae^{1x} + Be^{-1x} -2x^2 + x - 1
	\]
\end{document}