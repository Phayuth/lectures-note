\chapter{Mobile Robot Dynamic}
Mobile robot is a dynamics system. Using solely the kinematic model of the system is not enough to represent the system as a whole. For the robustness of the robot, the dynamics properties of the system such as external force, mass, inertia are considered.

\section{Differential Drive Mobile Robot Dynamic}

\begin{tcolorbox}[title=Lagrangain Formula,breakable]
	The dynamics model of the robot with constraint is derived using Lagrange formulation:
	\begin{equation}{\label{eq:lagrangain eq full}}
		\frac{\partial}{\partial t}(\frac{\partial \mathcal{L}}{\partial \Dot{q}_k})- \frac{\partial \mathcal{L}}{\partial q_k} + \frac{\partial P}{\partial \Dot{q}_k} + g_k + \tau_{dk}= f_k - \sum_{j=1}^{m}\lambda_j a_{jk}
	\end{equation}
	Where:
	\begin{itemize}
		\item {\makebox[1cm]{\(\mathcal{L}\)\hfill} is the Lagrangian}
		\item {\makebox[1cm]{\(P\)\hfill} is the power dissipation function due to friction and damping}
		\item {\makebox[1cm]{\(g_k\)\hfill} are the forces due to gravitation}
		\item {\makebox[1cm]{\(\tau_{dk}\)\hfill} are the system disturbances}
		\item {\makebox[1cm]{\(f_k\)\hfill} are the general forces (external influences to the system)}
		\item {\makebox[1cm]{\(q_k\)\hfill} general coordinate k=(1,...,n)}
		\item {\makebox[1cm]{m\hfill} is the number of linearly independent motion constraint}
		\item {\makebox[1cm]{\(\lambda_j\)\hfill} is the Lagrange multiplier associated with the jth constraint relation}
		\item {\makebox[1cm]{\(a_{jk}\)\hfill} is coefficients of the constraints (j=1,...,n)}
	\end{itemize}
	\break
	Assumption:
	\begin{itemize}
		\item {\makebox[1cm]{\(\mathcal{W}_p\)\hfill} = 0, planar robot where the potential energy is constant}
		\item {\makebox[1cm]{\(gk\)\hfill} = 0, planar robot where the potential energy is constant}
		\item {\makebox[1cm]{\(\tau_{dk}\)\hfill} = 0, no outside disturbances}
	\end{itemize}
\end{tcolorbox}


The Lagrangian \(\mathcal{L}\) is the difference between kinetic energy and potential energy. We get:
\begin{equation}
	\mathcal{L} = \mathcal{W}_k - \mathcal{W}_p 
\end{equation}
The Kinetic energy equation is:
\begin{equation}
	\mathcal{W}_k = \frac{1}{2}mV^2 + \frac{1}{2} J\Dot{\theta}^2
\end{equation}
The velocity in the 2D plane is:
\begin{equation}
	V^2 = \Dot{x}^2 + \Dot{y}^2
\end{equation}
We get the Lagrangian \(\mathcal{L}\):
\begin{equation}
	\mathcal{L} = \frac{m}{2}(\Dot{x}^2 + \Dot{y}^2) + \frac{J}{2}\Dot{\theta}^2
\end{equation}
Substitute back to \autoref{eq:lagrangain eq full}, we get:
\[\frac{\partial}{\partial t}(\frac{\partial \mathcal{L}}{\partial \Dot{x}}) = m\Ddot{x}\] 
\[\frac{\partial}{\partial t}(\frac{\partial \mathcal{L}}{\partial \Dot{y}}) = m\Ddot{y}\] 
\[\frac{\partial}{\partial t}(\frac{\partial \mathcal{L}}{\partial \Dot{\theta}}) = J\Ddot{\theta}\]
\[\frac{\partial \mathcal{L}}{\partial x} = 0\] 
\[\frac{\partial \mathcal{L}}{\partial y} = 0\] 
\[\frac{\partial \mathcal{L}}{\partial \theta} = 0\] 
The mobile robot constraint in x-axis is \(-sin\theta\), in y-axis is \(cos\theta\).\\
From \autoref{eq:lagrangain eq full}, we obtain:
\begin{equation}
	\begin{split}
		m\Ddot{x} = F_x - \lambda_1(-sin\theta) \\
		m\Ddot{y} = F_y - \lambda_1(cos\theta)\\
		J\Ddot{\theta} = M_z
	\end{split}
\end{equation}
We get:
\begin{equation}
	\begin{split}
		m\Ddot{x} - \lambda_1 sin \theta= F_x\\
		m\Ddot{y} + \lambda_1 cos \theta= F_y\\
		J\Ddot{\theta} = M_z
	\end{split}
\end{equation}
Since the assumption of no outside disturbance force, the force acting on the robot are the left wheel force \(F_l\) and the right wheel force \(F_r\). The resultant for is:
\begin{equation}
	F = F_r + F_l
\end{equation}
We have:
\begin{equation}
	\begin{split}
		F_r = \frac{\tau_r}{r}\\
		F_l = \frac{\tau_l}{r}
	\end{split}
\end{equation}
And
\begin{equation}
	\begin{split}
		F_x &= F cos\theta \\
		F_y &= F sin\theta \\
		M_z &= F_r \frac{l}{2} - F_l \frac{l}{2}
	\end{split}
\end{equation}
We get:
\begin{equation}
	\begin{split}
		F_x &= \frac{1}{r} (\tau_r + \tau_l) cos \theta \\
		F_y &= \frac{1}{r} (\tau_r + \tau_l) sin \theta \\
		M_z &= \frac{l}{2r} (\tau_r - \tau_l)
	\end{split}
\end{equation}
We get:
\begin{equation}
	\begin{split}
		m\Ddot{x} - \lambda sin \theta - \frac{1}{r} (\tau_r + \tau_l) cos \theta = 0\\
		m\Ddot{y} + \lambda cos \theta - \frac{1}{r} (\tau_r + \tau_l) sin \theta = 0\\
		J\Ddot{\theta} - \frac{l}{2r} (\tau_r - \tau_l) = 0
	\end{split}
\end{equation}
Rewrite the equation into Matrix form of:
\[M(q)\Ddot{q} + V (q,\Dot{q}) + F(\Dot{q}) = E(q)u - A^T(q) \lambda\]
We get:
\[M=\begin{bmatrix}
	m & 0 & 0\\
	0 & m & 0\\ 
	0 & 0 & J
\end{bmatrix},
E=\frac{1}{r}
\begin{bmatrix}
	cos\theta & cos\theta \\
	sin\theta & sin\theta \\ 
	\frac{l}{2} & -\frac{l}{2} 
\end{bmatrix},
A=\begin{bmatrix}
	-sin\theta & cos\theta & 0
\end{bmatrix},
u=\begin{bmatrix}
	\tau_r\\
	\tau_l 
\end{bmatrix}\]
Where remain are Zero.

\begin{tcolorbox}[title=Dynamic Model]
The model is
\begin{equation}
	\begin{bmatrix}
		\Dot{x}\\
		\Dot{y}\\
		\Dot{\theta}\\
		\Dot{V}\\
		\Dot{\omega}
	\end{bmatrix}=
	\begin{bmatrix}
		V cos\theta\\
		V sin\theta\\
		\omega\\
		0\\
		0
	\end{bmatrix}+
	\begin{bmatrix}
		0& 0 \\
		0& 0 \\
		0& 0 \\
		\frac{1}{mr}& \frac{1}{mr} \\
		\frac{l}{2Jr}&-\frac{l}{2Jr}
	\end{bmatrix}\begin{bmatrix}
		\tau_r\\
		\tau_l
	\end{bmatrix}
\end{equation}
\begin{equation}
	\begin{bmatrix}
		\tau_r \\
		\tau_l
	\end{bmatrix}=
	\begin{bmatrix}
		\frac{\dot{v}mr}{2} + \frac{\dot{\omega}Jr}{l} \\
		\frac{\dot{v}mr}{2} - \frac{\dot{\omega}Jr}{l}
	\end{bmatrix}
\end{equation}
\end{tcolorbox}