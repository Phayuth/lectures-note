\chapter{Linear Approximation with Taylor Series}
What is a Linear System ?
Linear System is a system that comply to 2 rules.
\begin{itemize}
	\item Superposition (Addition).
	\item Homogeneous (Multiplication). 
\end{itemize}

\section{Superposition}


Given that we have a function \(y = f(x)\).
\begin{itemize}
	\item If we have a value \(x_1\) substitute to the function we get \(y_1\) : \(y_1 = f(x_1)\)
	\item If we have a value \(x_2\) substitute to the function we get \(y_1\) : \(y_2 = f(x_2)\).
	\item If we have a value \(x_1 + x_2\) substitute to the function we should get \(y_1+y_2\) : \(y_1+y_2 = f(x_1+x_2)\)
\end{itemize}

\section{Homogeneous}

Given that we have a function \(y = f(x)\).
\begin{itemize}
	\item If we have a value \(\alpha x_1\) substitute to the function we get \(y_1\) : \(y_1 = f(\alpha x_1)\)
	\item If we have a value \(x_1\) substitute to the function then multiply by \(\alpha\) we should get \(y_1 = \alpha f(x_1)\)
\end{itemize}

\paragraph{Example} Find out if the function is linear : \(y = x\)
\paragraph{Superposition test} \(y_1 = x_1, y_2 = x_2\)
Add both result together \(y_1 + y_2 = x_1 + x_2\) \\
Substitute \(x_1 + x_2\) to the function we get \(y_1+y_2\). Thus, \(y_1+y_2=y_{12}\). TEST PASS.

\paragraph{Homogeneous test} Substitute \(\alpha x\) we get \(y = \alpha x\).
Substitute \(x\) and multiply by \(\alpha\) we get \(y = \alpha x\). Thus, \(\alpha x=\alpha x\). TEST PASS. Both test is passed and thus the system is linear.


\paragraph{Example} Find out if the function is linear : \(y = x^2\)
\paragraph{Superposition test}\(y_1 = x_1^2, y_2 = x_2^2\). Add both result together \(y_1 + y_2 = x_1^2 + x_2^2\). Substitute \(x_1 + x_2\) to the function we get \((x_1+x_2)^2\). Thus, \(y_1+y_2!=y_{12}\). TEST FAIL. The test is failed and thus the system is nonlinear.

\section{Linearization Process}
One of the Linearization method is by using Tyler Expansion Series within an operational range for stability.
\[
y \approx y(x_0) + \left[\frac{dy}{dx}|_{x_0}\frac{(x-x_0)}{1!}\right] + \left[\frac{d^2y}{dx^2}|_{x_0}\frac{(x-x_0)^2}{2!}\right] + ... [Higher Order Term]
\]
Let take a look at the plot:

\begin{figure}[h]
\centering
%		\def\svgscale{1}
\includesvg{src/tyler/tyler}
\caption{Mass spring system}
\label{fig:Plot}
\end{figure}

\(y=L(x)\) is the linear approximation of \(y=f(x)\) and \(a = x_0\) is an equilibrium point. We can see that we want to pick an operational range where the function is stable because the \(y=L(x)\) is close to \(y=f(x)\). As we move a way from the operational range, the approximation is starting to diverge from the real solution.


\paragraph{Example} Linearize : \(y = x^2\).
We have:
\[
y \approx y(x_0) + \left[\frac{dy}{dx}|_{x_0}\frac{(x-x_0)}{1!}\right] + \left[\frac{d^2y}{dx^2}|_{x_0}\frac{(x-x_0)^2}{2!}\right] + ... [Higher Order Term]
\]
Only consider the first order term and eliminate HOT because in HOT the variable \(x\) is subject to power number that will make it nonlinear. We get:
\[
y \approx y(x_0) + \left[\frac{dy}{dx}|_{x_0}\frac{(x-x_0)}{1!}\right]
\]
We get:
\[
\frac{dy}{dx}|_{x_0} = \frac{d(x^2)}{dx}|_{x_0} = 2x|_{x_0} = 2x_0
\]
We get:
\[
\begin{split}
	y &\approx y(x_0) + \left[2x_0\frac{(x-x_0)}{1!}\right]\\
	y &\approx y(x_0) + \left[2x_0(x-x_0)\right]
\end{split}
\]
\[
\boxed{y \approx y(x_0) + 2x_0x-2x_0^2}
\]
Let pick an equilibrium point \(x_0=2\)
\[
\begin{split}
	y &= 2^2 + 2\times2x-2\times2^2 \\
	y &= 4+4x-8 \\
	y &= 4x-4
\end{split}
\]
Now that we have a original function \(y=x^2\) and approximation function at \(x_0 = 2\) \(y=4x-4\). Let compare:
\[
\begin{split}
	x &= 2 \\
	=>y_{ori} &= 2^2 = 4 \\
	=>y_{lin} &= 4\times2 - 4 = 4
\end{split}
\]
Both are equal to each other at equilibrium point.
\[
\begin{split}
	x &= 3 \\
	=>y_{ori} &= 3^2 = 9 \\
	=>y_{lin} &= 4\times3 - 4 = 8
\end{split}
\]
A way from the equilibrium point, it starts to diverge.