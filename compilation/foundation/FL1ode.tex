\chapter{1st Order Differential Equation}
Differential equation is set an equation that its solution is a function and involve of its derivative. In engineering, these equations is usually used to govern a dynamics system model and the rate of change of state. In 1st Order Differential Equation is equation consist of first derivative of function in form:
\[
\frac{dy}{dx} + P(x)y = Q(x)
\]
Where :
\begin{itemize}
	\item {\makebox[1cm]{\(P(x)\)\hfill} is function of x}
	\item {\makebox[1cm]{\(Q(x)\)\hfill} is function of x}
\end{itemize}

\section{Method of Solving 1st Order Homogeneous Differential Equation}
\subsection{Method of Separation of Variable}
\begin{tcolorbox}[title=Method]
	\paragraph{When to use the Method}
	All ${y,dy}$ term and ${x,dx}$ can explicitly move to different side of the equation. For example:
	\[
	\frac{dy}{dx} = 5xy
	\]
	\[
	\rightarrow \frac{dy}{y} = 5xdx
	\]
	\paragraph{How to use the Method}
	\begin{itemize}
		\item Step 1 : Move all ${y,dy}$ term and ${x,dx}$ to different side of the equation.
		\item Step 2 : Integrate both side with respect to $ dx $ and $ dy $ respectively.
		\item Step 3 : Simplify the equation.
	\end{itemize}
\end{tcolorbox}

\paragraph{Example} Solve:
\[
\frac{dy}{dx} = 5xy
\]
\begin{itemize}
	\item Step 1
\end{itemize}
\[
\frac{dy}{y} = 5xdx
\]
\begin{itemize}
	\item Step 2
\end{itemize}
\[
\begin{split}
	\int \frac{1}{y} \,dy &= 5 \int x \,dx \\
	ln|y| &= \frac{5}{2} x^2 + c
\end{split}
\]
\begin{itemize}
	\item Step 3
\end{itemize}
\[
\begin{split}
	ln|y| &= \frac{5}{2} x^2 + c \\
	e^{ln|y|} &= e^{\frac{5}{2} x^2 + c}\\
	y &= e^{\frac{5}{2} x^2 + c}\\
	y &= e^{\frac{5}{2} x^2}e^{c}\\
	y &= Ce^{\frac{5}{2} x^2} \leftarrow Solution\\ 
\end{split}
\]


\section{Method of Solving 1st Order Non-Homogeneous Differential Equation}
\subsection{Method of Variable Substitution}
\begin{tcolorbox}[title=Method]
	\paragraph{When to use the Method} Use in general form of 1st order linear differential equation of:
	\[\frac{dy}{dx} + P(x)y = Q(x)\]
	\paragraph{How to use the Method}
	\begin{itemize}
		\item Step 1 : Substitute $ y = uv $ and $ \frac{dy}{dx} = u\frac{dv}{dx} + v \frac{du}{dx}$ to equation.
		\item Step 2 : Factoring $ v $ out. example: $ v(term\_u, term\_x) $.
		\item Step 3 : Put $ v $ term equal to zero and solve for $ u $ using separation of variable.
		\item Step 4 : Substitute $ u $ back to equation Step 2 where $ v $ term is zero and Solve for $ v $.
		\item Step 5 : After getting $ u $ and $v$, substitute back into $ y = uv $ for a solution of function.
	\end{itemize}
\end{tcolorbox}

\paragraph{Example} Solve:
\[
\frac{dy}{dx} - \frac{y}{x}= 1
\]
\begin{itemize}
	\item Step 1
\end{itemize}
\[
u\frac{dv}{dx} + v \frac{du}{dx} - \frac{uv}{x}= 1
\]
\begin{itemize}
	\item Step 2
\end{itemize}
\[
u\frac{dv}{dx} + v( \frac{du}{dx} - \frac{u}{x})= 1
\]
\begin{itemize}
	\item Step 3
\end{itemize}
\[
\begin{split}
	( \frac{du}{dx} - \frac{u}{x}) &= 0  \\
	\frac{du}{dx} &= \frac{u}{x} \\
	\frac{du}{u} &= \frac{dx}{x} \\
	\int \frac{du}{u} &= \int \frac{dx}{x} \\
	ln|u| &= ln|x| + C \\
	ln|u| &= ln|x| + ln|K| \leftarrow \text{let C = ln|k| make easier} \\
	u &= Kx
\end{split}
\]
\begin{itemize}
	\item Step 4
\end{itemize}
\[
\begin{split}
	Kx\frac{dv}{dx} &= 1 \\
	dv &= \frac{1}{Kx} dx \\
	\int dv &= \frac{1}{K}\int \frac{1}{x} dx \\
	v &= \frac{1}{K} (ln|x| + D) \\
	v &= \frac{1}{K} ln|Lx|
\end{split}
\]
\begin{itemize}
	\item Step 5
\end{itemize}
Our Solution is:
\[
y = uv = Kx\frac{1}{K} ln|Lx| = xln|Lx|
\]


\subsection{Method of Integrating Factor}
\begin{tcolorbox}[title=Method]
	Use in general form of 1st order linear differential equation of:
	\[\frac{dy}{dx} + P(x)y = Q(x)\]
	\paragraph{How to use the Method}
	\begin{itemize}
		\item Step 1 : Calculate Integrating Factor $ I(x) = e^{\int P(x)dx}$.
		\item Step 2 : Multiply both side of the equation by $ I(x) $
		\item Step 3 : Form $ \frac{d}{dx} (y.I(x)) = I(x)Q(x)$ and Integrate both side by $ dx $.
		\item Step 4 : Solve for $ y $ and simplify.
	\end{itemize}
\end{tcolorbox}

\paragraph{Example} Solve:
\[
cos(x)\frac{dy}{dx} + sin(x)y= 1
\]
Then :
\[
\frac{dy}{dx} + tan(x)y= \frac{1}{cos(x)}
\]
We have $ P(x) = tan(x)$ and $ Q(x) = \frac{1}{cos(x)} $
\begin{itemize}
	\item Step 1
\end{itemize}
\[
I(x) = e^{\int P(x)dx} = e^{\int tan(x)dx} = e^{ln|sec(x)|} = sec(x)
\]
\begin{itemize}
	\item Step 2
\end{itemize}
\[
sec(x)\frac{dy}{dx} + sec(x)tan(x)y= sec(x)\frac{1}{cos(x)}
\]
\[
sec(x)\frac{dy}{dx} + sec(x)tan(x)y= sec^2(x)
\]
\begin{itemize}
	\item Step 3
\end{itemize}
\[
\begin{split}
	\frac{d}{dx}(y.sec(x)) &= sec^2(x) \\
	\int \frac{d}{dx}(y.sec(x))dx &= \int sec^2(x) dx\\
	y.sec(x) &= \int sec^2(x) dx\\
	y.sec(x) &= tan(x) + C\\
\end{split}
\]
\begin{itemize}
	\item Step 4
\end{itemize}
\[
\begin{split}
	y.sec(x) &= tan(x) + C\\
	y &= \frac{tan(x) + C}{sec(x)} \\
	y &= sin(x) + Ccos(x)
\end{split}
\]