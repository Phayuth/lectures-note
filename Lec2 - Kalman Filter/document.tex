\documentclass[12pt,a4paper]{article}
\usepackage[utf8]{inputenc}
\usepackage[T1]{fontenc}
\usepackage{amsmath}
\usepackage{amsfonts}
\usepackage{amssymb}
\usepackage{graphicx}
\usepackage[left=2.54cm, right=2.54cm, top=2.54cm, bottom=2.54cm]{geometry}
\usepackage{hyperref} %for reference automatically
\usepackage{fancyhdr}

\pagestyle{fancy} % for header and footer
\fancyhf{}
\fancyhead[LE,RO]{Prepared by: Phayuth}
\fancyhead[RE,LO]{Supervisor : Dr.Sarot Srang}
\fancyfoot[LE,RO]{Page \thepage}
\renewcommand{\headrulewidth}{2pt}
\renewcommand{\footrulewidth}{1pt} % for header and footer

% THIS IS THE XML CODE INCLUDE==============================================================
\usepackage{listings}
\usepackage{xcolor}
\definecolor{codegreen}{rgb}{0,0.6,0}
\definecolor{codegray}{rgb}{0.5,0.5,0.5}
\definecolor{codepurple}{rgb}{0.58,0,0.82}
\definecolor{backcolour}{rgb}{0.95,0.95,0.92}
\lstset{
	backgroundcolor=\color{backcolour},   
	commentstyle=\color{codegreen},
	keywordstyle=\color{magenta},
	numberstyle=\tiny\color{codegray},
	stringstyle=\color{codepurple},
	numbers=left,
	breaklines=true,
	tabsize=2,
	basicstyle=\ttfamily\footnotesize,
	literate={\ \ }{{\ }}1
}


\begin{document}
	\section*{\centering Lesson 2 : Kalman Filter}
	\section{Background}
		Kalman Filter was found by Dr. Rudolf Emil Kálmán. This algorithm is a powerful filtering algorithm that has been used in many applications most notably in signal processing, control, optimization, sensor fusion, system identification -etc and it is able to implement online.
	
	\section{Kalman Filter (Linear System)}
	
	\section{Extended Kalman Filter (Nonlinear System)}
	
	\section{Unscented Kalman Filter (Nonlinear System)}
		
\end{document}